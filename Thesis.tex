%%%%%%%%%%%%%%%%%%%%%%%%%%%%%%%%%%%%%%%%%%%%%%%%%%%%%%%%%%%%%%%%%%%%
%% I, the copyright holder of this work, release this work into the
%% public domain. This applies worldwide. In some countries this may
%% not be legally possible; if so: I grant anyone the right to use
%% this work for any purpose, without any conditions, unless such
%% conditions are required by law.
%%%%%%%%%%%%%%%%%%%%%%%%%%%%%%%%%%%%%%%%%%%%%%%%%%%%%%%%%%%%%%%%%%%%

\documentclass[
  printed, %% This option enables the default options for the
           %% digital version of a document. Replace with `printed`
           %% to enable the default options for the printed version
           %% of a document.
  table,   %% Causes the coloring of tables. Replace with `notable`
           %% to restore plain tables.
  nolof,     %% Prints the List of Figures. Replace with `nolof` to
           %% hide the List of Figures.
  nolot,     %% Prints the List of Tables. Replace with `nolot` to
           %% hide the List of Tables.
           oneside, color
  %% More options are listed in the user guide at
  %% <http://mirrors.ctan.org/macros/latex/contrib/fithesis/guide/mu/fi.pdf>.
]{fithesis3}
%% The following section sets up the locales used in the thesis.
\usepackage[resetfonts]{cmap} %% We need to load the T2A font encoding
\usepackage[T1,T2A]{fontenc}  %% to use the Cyrillic fonts with Russian texts.
\usepackage[
  main=czech, %% By using `czech` or `slovak` as the main locale
                %% instead of `english`, you can typeset the thesis
                %% in either Czech or Slovak, respectively.
  german, russian, czech, slovak, english %% The additional keys allow
]{babel}        %% foreign texts to be typeset as follows:
%%
%%   \begin{otherlanguage}{german}  ... \end{otherlanguage}
%%   \begin{otherlanguage}{russian} ... \end{otherlanguage}
%%   \begin{otherlanguage}{czech}   ... \end{otherlanguage}
%%   \begin{otherlanguage}{slovak}  ... \end{otherlanguage}
%%
%% For non-Latin scripts, it may be necessary to load additional
%% fonts:
\usepackage{paratype}
\def\textrussian#1{{\usefont{T2A}{PTSerif-TLF}{m}{rm}#1}}
%%
%% The following section sets up the metadata of the thesis.
\thesissetup{
    date          = \the\year/\the\month/\the\day,
    university    = mu,
    faculty       = fi,
    type          = mgr,
    author        = Bc. Radim Lipovčan,
    gender        = m,
    advisor       = RNDr. Vlasta Šťavová,
    title         = {Používání a získávání kryptoměny Monero z pohledu použitelné bezpečnosti},
    TeXtitle      = {Používání a získávání kryptoměny Monero z pohledu použitelné bezpečnosti},
    keywords      = {Monero, usable security, cryptocurrency, mining, pool, blockchain},
    TeXkeywords   = {Monero, usable security, cryptocurrency, mining, pool, blockchain},
}
\thesislong{abstract}{
    This is the abstract of my thesis, which can

    span multiple paragraphs.
}
\thesislong{thanks}{
    This is the acknowledgement for my thesis, which can

    span multiple paragraphs.
}
%% The following section sets up the bibliography.
\usepackage{csquotes}
\usepackage[              %% When typesetting the bibliography, the
  backend=biber,          %% `numeric` style will be used for the
  style=numeric,          %% entries and the `numeric-comp` style
  citestyle=numeric-comp, %% for the references to the entries. The
  sorting=none,           %% entries will be sorted in cite order.
  sortlocale=auto         %% For more unformation about the available
]{biblatex}               %% `style`s and `citestyles`, see:
%% <http://mirrors.ctan.org/macros/latex/contrib/biblatex/doc/biblatex.pdf>.
\addbibresource{Thesis.bib} %% The bibliograpic database within
                          %% the file `example.bib` will be used.
\usepackage{makeidx}      %% The `makeidx` package contains
\makeindex                %% helper commands for index typesetting.
%% These additional packages are used within the document:
\usepackage{paralist}
\usepackage{amsmath}
\usepackage{amsthm}
\usepackage{amsfonts}
\usepackage{url}
\usepackage{menukeys}

%packages
\usepackage{float} 
\begin{document}

\chapter{Úvod}
\chapter{Kryptoměna Monero}
Monero je opensource kryptoměnou, která je vyvíjená v rámci Monero projektu za účelem vytvoření decentralizované a anonymní měny. Hlavním cílem je dát uživateli plnou kontrolu nad předstředky, kterými disponuje. Znamná to tedy, že jednotlivé transakce a zůstatek nemohou být bez vědomí dohledány \cite{moneroprojectgithub}.
\section{Původ a zaměření kryptoměny}
Cesta vzniku samotného Monera je složená z několika forků a předcházejících projektů. Prvním z nich byl Bytecoin jakožto proof-of-concept kryptoměna vzniklá implementací protokolu CryptoNote, který byl publikován na začátku roku 2014 \cite{githubbytecoin}.
% CryptoNote implementace ByteCoin -> https://github.com/monero-project/monero/commit/296ae46ed8f8f6e5f986f978febad302e3df231a
Bytecoin měl ovšem problém s tzv. preminem, tedy stavem, kdy kryptoměna publikovaná na veřejnost má již velkou část peněz předem vytěženou \cite{}
% Duvod forku https://coinmetrics.io/papers/dissertation.pdf premine xx procent
% Fork Bytecoin -> BitMonero https://github.com/monero-project/monero/commit/1a8f5ce89a990e54ec757affff01f27d449640bc

% Monero attack https://lab.getmonero.org/pubs/MRL-0002.pdf
 
V roce 2014 došlo k založení projektu Monero jako měny, která je postavená na CryptoNote protokolu s cílem přinést škálovatelnou, decentralizovanou a soukromí ochraňující digitální měnu \cite{monerohistory}. 
\newpage
 
\subsection{Současný vývoj}
\subsection{Alternativy}
\section{Blockchain technologie}
\subsection{Block}
\subsection{Návaznost a provázanost bloků}
\section{CryptoNote protokol}
\subsection{Verzování a aktualizace}
\subsection{Proof-of-Work}
\section{Prvky sítě}
\subsection{Peněženky}
\subsection{Nodes}
\subsection{Miners}
\subsection{Pools}
\section{Adresování v síti}
\subsection{Typy adres}
\subsection{Transakční adresa}
\subsection{View klíč}
\section{Transakce}
\subsection{Struktura}
\subsection{Vstupy, výstupy, mixing}
\subsection{Průběh transakce}
\subsection{Sledování transakcí na síti}


\chapter{Používání monera}
\section{Pěněženky}
\subsection{Hot wallet}
\subsection{Cold wallet}
\subsection{View only wallet}
\subsection{Možnosti útoků na peněženku}
\subsection{Přehled bezpečnosti úložných způsobů}
\section{Software pro práci s kryptopenězi}
\subsection{Vytvoření peněženky}
\subsection{Fullnode, remote node}
\subsection{Multisig}
\subsection{Srovnání wallet softwaru}
\subsection{Exchanges}
\section{Scamy v prostředí kryptoměny}
\subsection{Online portály}
\subsection{Mining pooly}
\subsection{Cílený malware}
\subsection{Delivery chain}

\section{Monero usecase}
\subsection{Darknet Markety}
\subsection{Monero Markety}
\subsection{Úskalí tradingu anonymní měny}
\subsection{Srovnání s ostatními kryptoměnami}

\section{Anonymita při užívání}

\chapter{Výzkum uživatelů kryptoměny}
\section{Definice, výzkumné otázky}
\section{Na jakém vzorku}
\section{Vyhodnocení international, CZ}

\chapter{Best practices pro usage a storage Monera}
\section{Návrh bezpečného úložného systému}
\section{Návrh pro používání kryptoměny}
\section{Portál nabízející služby pro bezpečné používání kryptoměny}



\chapter{Získávání Monera a zajištění chodu sítě}
\section{PoW Monera}
\section{Mining pooly, solo mining}
\section{Web mining, botnet mining}
\section{Cloud mining}
\section{Systémy pro těžbu - ASIC, dodávané}
\section{Software používaný k těžbě}
\section{Srovnání způsobů pro získávání kryptoměny}

\chapter{Průzkum způsobů těžby}
\section{Definice, výzkumné otázky}
\section{Na jakém vzorku}
\section{Vyhodnocení international, CZ}



\chapter{Průzkum operátorů sítě}
\section{Definice, výzkumné otázky}
\section{Na jakém vzorku}
\section{Vyhodnocení international, CZ}

 
\chapter{Mining malware}
\section{Prevence, detekce a recovery}
\subsection{Server část}
\subsection{Běžní uživatelé}


\chapter{Návrh bezpečného těžebního systému}
\section{Linux-based řešení}
Ansible, Centos 7
\section{Windows-based řešení}
Windows 10 ISO unattended install, Powershell scripty, stažení a instalace Stak XMR

\chapter{Plán}

\begin{figure}[H]
\center
\begin{tabular}{ll}
Měsíc     & Task                                              \\
1.7.2018  & Kryptoměna Monero; Používání Monera               \\
1.8.2018  & Návrh výzkumu uživatelů;Průzkumu těžby;Operátorů  \\
1.9.2018  & Sběr dat; Získávání Monera a zajištění chodu sítě \\
1.10.2018 & Vyhodnocení dat                                   \\
1.11.2018 & Best practices pro usage a storage                \\
1.12.2018 & Mining malware; Vytvoření těžebního systému       \\
1.1.2019  & Cryptocore.cz jako výstup poznatků                \\
1.2.2019  & Rezerva                                           \\
1.3.2019  & Měsíc na odladění                                 \\
1.4.2019  & Finální verze + tisk                             
\end{tabular}
\caption{Plán práce na diplomce}
\label{ssme-thesis-plan}
\end{figure}



\printbibliography[heading=bibintoc]

%\appendix %% Start the appendices.
%\chapter{Příloha}
%Monero
\end{document}
